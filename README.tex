\documentclass{article}
\usepackage{graphicx}
\begin{document}
\title{Script for cross-corner simulations}
\author{Rakesh K K\\rakeshkk@ee.iitb.ac.in}
\maketitle
This document gives an overview of the script which is written to simplify the work of simulationg cross-corner simulation in cadence design environment. 

\section{Using the script}
\begin{itemize}
\item Copy the scripts simulate\_regress.il and simulate\_regress.pl to directory of your choice. 
    	For simplicity, you can shoose this directory to be the one where virtuoso/icfb is invoked.
\item In the directory where virtuoso/icfb is invoked, open the file .cdsinit and add the line \texttt{load "<PATH of simulate\_regress.il file>"} to it.
\item First, run a simple simulation using ADE. So that netlist files get created. 
    	This netlist file is then used by the script for cross-corner simulations
\item Go to the schematic window and press the keys \textbf{"Ctrl+Q"}. This should pop up a gedit(text editor) window with some pre-written lines it. 
    	Pre-written lines are the syntax of the input file for the cross-corner simulation script. It looks as below.\\
	\texttt{\#File created by the script simulate\_regress.il\\
	\#Mail to rakeshkk@ee.iitb.ac.in\\
	\#corners\\
	\#(SS,SS\_res,TT\_cap),120\\
	\#(TT,FF\_res,FF\_cap),000\\
	\#end\_corners\\
	\#process\\
	(SS,SS\_res,TT\_cap)\\
	(TT,FF\_res,FF\_cap)\\
	\#end\_process\\
	\#temp\\
	80\\
	40\\
	\#end\_temp}\\
	In the file "\#" implies a comment. If any old line is to be commented, add a \# infront of that line. 
	However, \# infront of the lines with the keywords 'corners', 'process' and 'temp' are for the scripting convinience.
\item Once necessary editing is done in the file which is open, it can be closed. 
\item On closing the text editor, a terminal window pops up. Simulations start running.	
\item Once the simulations are done, they can be viewed using waveform viewer. 
    	The simulation results will be saved in the same hieracrchy as before with the directory name as \textbf{cross\_corner}.
\end{itemize}

\section{Details about the script}
Script is mainly divided into two parts. viz., SKILL and PERL. 
SKILL part is to interface the script with the schematic (cadence environment). 
PERL part of the script will create the directory structure and then simulate.
\subsection{SKILL script}
The file name of the SKILL script is simulate\_regress.il. 
This file creates the input file which is required by the PERL script. 

The script can be executed by clicking the keys \textbf{"Ctrl+Q"} in the schematic window. 
Please note that before executing this, a simple simulation must be run using ADE so that netlist is already created. 

On pressing the keys, a text editor window will open with the syntax prewritten in the file. 
While editing this file, please make sure that the name of the cornes are you have given are valid ones. 
It is also to be noted that, unless the opened text file is closed, Cadence will not respond. 
Once the file is closed, cadence will start responding. 
Also, on closing the file, SKILL executes PERL script in a newly opened terminal. 


\subsection{PERL script}
The file name of the PERL script is simulate\_regress.pl. The file is to be saved in the same directory as that of the simulate\_regress.il.

Following is the sequence of the functions by PERL script.
\begin{itemize}
\item The script deletes older simulation data, if any. 
\item It checks the input file and creates the directories according to the number of combinations in the input file.
\item From the previously run simulation(using ADE) output data, the netlist file is taken. This netlist is copied to all the new directories created.
\item Once the netlist file is copied, the process corners and temperatures are modified in the respective folders.
\item Simulations are run after modifications. 
\end{itemize}

\subsection{Details about input file}
There are currently 3 sections allowed in the input file. \\
The allowed sections are corners, process and temp. Corners is the keyword where the process-temp combination can be given. Syntax of the file should be as in fig.\ref{fig:input_file}. 

If all the lines in the section 'corners' are commented, then the script considers the other sections process and temp. 
When they are considered, the simulations will be run for all the combinations of process and temp. \\

\begin{figure}[htp]
\centering
\includegraphics[scale=0.60]{/home/rakesh/scripts/cross_corner_sim/corner_info_file.jpg}
\caption{Screenshot of the input file which is to be filled for cross-corner simulations}
\label{fig:input_file}
\end{figure}
\begin{figure}[htp]
\centering
\includegraphics[scale=0.60]{/home/rakesh/scripts/cross_corner_sim/konsole.jpg}
\caption{Screesnhot of the terminal which opens once the PERL script starts running after the text editor is closed}
\label{}
\end{figure}
\end{document}